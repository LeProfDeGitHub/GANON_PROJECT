\section{Projet STM}

Le programme de l'OMEGAAA est écrit à l'aide de l'environnement de développement
STMicroelectronics. Ce dernier permet de créer des projets pour les microcontrôleurs
de la firme. Chaque projets est composé de plusieurs sous-dossiers et fichiers. Le
HAL\footnote{HAL (pour \textbf{H}ardware \textbf{A}bstraction \textbf{L}ayer) est
une bibliothèque logicielle qui fournit une interface de programmation pour les
périphériques matériels.} fournit par STMicroelectronics étant écrit en C, et que
bons nombres d'information, tutoriels et exemples sont disponibles en C, le programme
de l'OMEGAAA utilisera ce language.


\begin{minipage}{0.95\textwidth}
    \begin{wrapfigure}{L}{0.35\textwidth}
    \vspace{-1.0cm}
    \dirtree{%
    .1 ....
    .1 Core.
        .2 Inc.
            .3 ....
        .2 Src.
            .3 drivers.
                .4 BMI088.c.
                .4 BMP388.c.
                .4 buzzer.c.
                .4 gps.c.
                .4 rfm96w.c.
                .4 w25q\_mem.c.
                .4 ....
            .3 peripherals.
                .4 dma.c.
                .4 gpio.c.
                .4 i2c.c.
                .4 spi.c.
                .4 tim.c.
                .4 usart.c.
                .4 ....
            .3 flash\_stream.c.
            .3 main.c.
            .3 scheduler.c.
            .3 tools.c.
            .3 ....
            .3 stm32f4xx\_hal\_msp.c.
            .3 stm32f4xx\_it.c.
            .3 syscalls.c.
            .3 sysmem.c.
            .3 system\_stm32f4xx.c.
    .1 ....
    }
    \caption{Arboressance du programme}
    \end{wrapfigure}
    
    \vspace{1.0cm}
    
    L'intégralité du programme de l'OMEGAAA se trouve dans le dossier
    "\texttt{Core}". Ce dernier est composé des fichiers de déclarations (.h) dans
    le dossier \texttt{Inc} et des fichiers de programme (.c) dans le dossier
    \texttt{Src}. Tous     les programmes relatifs aux composants externes au
    microcontrôleur (capteurs, ...) se trouve dans le dossier \texttt{drivers} et
    ceux relatifs à la gestion des composants internes (gestionnaires des
    périphériques, ...) dans \texttt{peripherals}.
    
    \vspace{0.5cm}
    
    On retrouve dans le dossier \texttt{Inc} symétriquement la même arboressance de
    dossiers et fichiers que dans le dossier \texttt{Src} à l'extention près (non
    des "\texttt{.c}" mais des "\texttt{.h}").
    
    \vspace{0.5cm}
    
    Le point d'entrée du programme se trouve dans la fonction \texttt{main} du
    fichier du même nom.
    
    
    \vspace{0.2cm}
    Le fichier \texttt{scheduler.c} contient l'implémentation de l'ordonanceur et
    la définition d'une tâche (cf. \ref{subsec:implementation}).

    
    \vspace{0.2cm}
    Le fichier \texttt{flash\_stream.c} contient les structures et fonctions
    permettant de lire et d'écrire des données dans une mémoire tout en gérant
    les pointeurs de lecture et d'écriture.

    
    \vspace{0.2cm}
    Le fichier \texttt{tools.c} contient des fonctions utilisées à différents
    endroits du programme.

    \vspace{0.5cm}

    Enfin, les fichiers \texttt{stm32f4xx\_hal\_msp.c},
    \texttt{stm32f4xx\_it.c}, \texttt{syscalls.c}, \texttt{sysmem.c} et
    \texttt{system\_stm32f4xx.c} sont des fichiers de configuration,
    de gestion des interruptions, de gestion de la mémoire et de gestion
    du système général du microcontrôleur. Ces fichiers sont générés
    automatiquement par l'environnement de développement et bien qu'ils le
    peuvent, ils ne seront pas modifiés par l'utilisateur.
    
\end{minipage}