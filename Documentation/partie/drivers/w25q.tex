\subsection{Mémoire Flash W25Q\_XXX}
\label{subsec:w25q}

La mémoire flash W25Q\_XXX est un composant de stockage de données non volatiles
de la marque Winbond. Elle est utilisée dans le projet OMEGAAA pour stocker les
données acquises par les capteurs. La mémoire flash W25Q\_XXX utilise une interface
SPI pour communiquer avec le microcontrôleur. Elle est composé de plusieurs blocs
de mémoire appelés secteurs. Chaque secteur est composé de 16 pages de 256 octets,
soit 4 Ko par secteur. La mémoire flash W25Q\_XXX dispose de plusieurs commandes
permettant nottamant de lire, écrire, effacer des données, de connaitre et modifier
l'état du composant.\\

Les fichiers correspondant à la gestion de la mémoire flash W25Q\_XXX sont les
suivants :

\begin{itemize}
    \item \texttt{w25q\_mem.h} : Déclaration des structures et des fonctions
    \item \texttt{w25q\_mem.c} : Implémentation des fonctions
\end{itemize}

\subsubsection{Déclaration des structures et des fonctions}

\subsubsubsection*{\texttt{W25Q\_Chip}}

La structure contient les informations nécessaires pour communiquer avec la
mémoire flash et pour gérer son état.

\begin{lstlisting}[style=prog, frame=shadowbox, label={lst:w25q_dec},
    emph={[1]W25Q_Init}, emphstyle={[1]\color{C}},
    emph={[2]GPIO_TypeDef, W25Q_Chip, SPI_HandleTypeDef_flag}, emphstyle={[2]\color{E}}]
typedef struct W25Q_Chip {
	SPI_HandleTypeDef_flag *hspi_flag;
	GPIO_TypeDef           *csPinBank;
	uint16_t                csPin;
	bool                    status_bits[24];
	bool                    ASYNC_busy;
} W25Q_Chip;

void W25Q_Init(W25Q_Chip              *w25q_chip,
               SPI_HandleTypeDef_flag *hspi_flag,
               GPIO_TypeDef           *csPinBank,
               uint16_t                csPin,
               uint32_t                id);
\end{lstlisting}

Les trois premiers champs de la structure \texttt{W25Q\_Chip} sont des informations
nécessaires pour la communication SPI.

Le tableau \texttt{status\_bits} contient les bits d'état de la mémoire flash. Ces
bits servent par exemple à indiquer si la mémoire flash est occupée par une
opération interne. En effet, lorsque la mémoir flash est en train d'écrire ou
d'effacer des données, elle ne peut pas effectuer d'autres opérations et ainsi bons
nombre de commandes envoyer ne seront pas pris en compte par le composant.

Le champ \texttt{ASYNC\_busy} est un booléen qui indique si la mémoire flash est
occupée par une opération asynchrone. Cela permet d'assurer que certaines tâches
asynchrones, puissent s'executer une par une et non pas en parallèle.

La fonction \texttt{W25Q\_Init} permet d'initialiser la structure \texttt{W25Q\_Chip}
avec les informations nécessaires pour communiquer avec la mémoire flash. Elle
prend en paramètre un pointeur vers la structure \texttt{W25Q\_Chip}, les informations
nécessaires pour la communication SPI ainsi que l'identifiant de la mémoire flash
W25Q\_XXX. L'identifiant est utilisé pour vérifier le bon fonctionnement de la
communication SPI avec la mémoire flash.

\newpage

\subsubsubsection*{\texttt{ASYNC\_W25Q\_ReadStatusReg} : }

La tâche \texttt{ASYNC\_W25Q\_ReadStatusReg} permet de lire les bits d'état de la
mémoire flash. La mémoire flash W25Q\_XXX disposant de trois registres d'état, cette
tâche envoie trois commandes de lecture pour récupérer les bits d'état de chacun des
registres. Elle mettera ensuite à jour le tableau \texttt{status\_bits} de la
structure \texttt{W25Q\_Chip} passée au préalable en paramètre.

\begin{lstlisting}[style=prog, frame=shadowbox, label={lst:ASYNC_W25Q_ReadStatusReg},
    emph={[1]ASYNC_W25Q_ReadStatusReg_init, ASYNC_W25Q_ReadStatusReg}, emphstyle={[1]\color{C}},
    emph={[2]W25Q_Chip, SCHEDULER, TASK}, emphstyle={[2]\color{E}}]
void ASYNC_W25Q_ReadStatusReg_init(TASK *task, W25Q_Chip *w25q_chip);
void ASYNC_W25Q_ReadStatusReg(SCHEDULER *scheduler, TASK *self);
\end{lstlisting}

La tâche se termine lorsque tous les bits d'état ont été lus et stockés dans le
tableau \texttt{status\_bits}.

% \vfill

\subsubsubsection*{\texttt{ASYNC\_W25Q\_WriteEnable} : }

La tâche \texttt{ASYNC\_W25Q\_WriteEnable} permet d'activer l'écriture dans la
mémoire flash. En effet, avant de pouvoir écrire ou d'effacer des données, il est
nécessaire qu'un certain bit d'état se trouvant dans un registre de la mémoire flash
soit activé. Cette tâche envoie la commande d'activation de l'écriture à la mémoire
flash.

\begin{lstlisting}[style=prog, frame=shadowbox, label={lst:ASYNC_W25Q_WriteEnable},
    emph={[1]ASYNC_W25Q_WriteEnable_init, ASYNC_W25Q_WriteEnable}, emphstyle={[1]\color{C}},
    emph={[2]W25Q_Chip, SCHEDULER, TASK}, emphstyle={[2]\color{E}}]
void ASYNC_W25Q_WriteEnable_init(TASK *task, W25Q_Chip *w25q_chip);
void ASYNC_W25Q_WriteEnable(SCHEDULER *scheduler, TASK *self);
\end{lstlisting}

La tâche se termine lorsque l'envoie de la commande d'activation a été programmé
dans le scheduler. Cependant, pendant l'exécution de la tâche, cette dernière
transfert le pointeur \texttt{self->is\_done} à la tâche se chargeant de l'envoie de
la commande. Ainsi, lorsque l'entité génératrice de la tâche
\texttt{ASYNC\_W25Q\_WriteEnable} détecte la fin de son exécution, cela revient à
détecter que la commande s'est rélement bien envoyé. (Cette technique est utilisé 
afin d'optimiser le nombre de tâche en cours d'exécution.) 

% \vfill

\subsubsubsection*{\texttt{ASYNC\_W25Q\_WaitForReady} : }

La tâche \texttt{ASYNC\_W25Q\_WaitForReady} permet d'attendre que la mémoire flash
soit prête à écrire ou à effacer des données. Elle appelle la tâche
\texttt{ASYNC\_W25Q\_ReadStatusReg} pour lire les bits d'état de la mémoire flash
jusqu'à ce que le bit d'état indiquant que la mémoire flash est occupée soit à 0.

\begin{lstlisting}[style=prog, frame=shadowbox, label={lst:ASYNC_W25Q_WaitForReady},
    emph={[1]ASYNC_W25Q_WaitForReady_init, ASYNC_W25Q_WaitForReady}, emphstyle={[1]\color{C}},
    emph={[2]W25Q_Chip, SCHEDULER, TASK}, emphstyle={[2]\color{E}}]
void ASYNC_W25Q_WaitForReady_init(TASK *task, W25Q_Chip *w25q_chip);
void ASYNC_W25Q_WaitForReady(SCHEDULER *scheduler, TASK *self);
\end{lstlisting}

La tâche se termine lorsque le bit d'état indiquant que la mémoire flash est occupée
est à 0.

% \vfill

\subsubsubsection*{\texttt{ASYNC\_W25Q\_EraseSectore} : }

La tâche \texttt{ASYNC\_W25Q\_EraseSectore} permet d'effacer un secteur de la mémoire
flash. Elle envoie la commande d'effacement d'un secteur à la mémoire flash. A noter
que la tâche attend d'abord que le booléen \texttt{ASYNC\_busy} de la structure
\texttt{W25Q\_Chip} soit faux avant de s'executer proprement et le met ensuite à vrai.

\begin{lstlisting}[style=prog, frame=shadowbox, label={lst:ASYNC_W25Q_EraseSectore},
    emph={[1]ASYNC_W25Q_EraseSectore_init, ASYNC_W25Q_EraseSectore}, emphstyle={[1]\color{C}},
    emph={[2]W25Q_Chip, SCHEDULER, TASK}, emphstyle={[2]\color{E}}]
void ASYNC_W25Q_EraseSectore_init(TASK *task, W25Q_Chip *w25q_chip, uint32_t addr);
void ASYNC_W25Q_EraseSectore(SCHEDULER *scheduler, TASK *self);
\end{lstlisting}

La tâche se termine lorsque l'envoie de la commande d'effacement s'est bien déroulé.
Elle mettra alors le booléen \texttt{ASYNC\_busy} à faux avant de disparaitre.

% \vfill

\subsubsubsection*{\texttt{ASYNC\_W25Q\_EraseAll} : }

La tâche \texttt{ASYNC\_W25Q\_EraseAll} permet d'effacer toute la mémoire flash. Elle
envoie la commande d'effacement de toute la mémoire flash à la mémoire flash. A noter
que la tâche attend d'abord que le booléen \texttt{ASYNC\_busy} de la structure
\texttt{W25Q\_Chip} soit faux avant de s'executer proprement et le met ensuite à vrai.
L'opération d'effacement de toute la mémoire flash est une opération longue qui peut
dépendant de la taille de la mémoire flash prendre plusieurs dizaines de secondes.

\begin{lstlisting}[style=prog, frame=shadowbox, label={lst:ASYNC_W25Q_EraseAll},
    emph={[1]ASYNC_W25Q_EraseAll_init, ASYNC_W25Q_EraseAll}, emphstyle={[1]\color{C}},
    emph={[2]W25Q_Chip, SCHEDULER, TASK}, emphstyle={[2]\color{E}}]
void ASYNC_W25Q_EraseAll_init(TASK *task, W25Q_Chip *w25q_chip);
void ASYNC_W25Q_EraseAll(SCHEDULER *scheduler, TASK *self);
\end{lstlisting}

La tâche se termine lorsque l'envoie de la commande d'effacement s'est bien déroulé.
Elle mettra alors le booléen \texttt{ASYNC\_busy} à faux avant de disparaitre.

\subsubsubsection*{\texttt{ASYNC\_W25Q\_ReadData} : }

La tâche \texttt{ASYNC\_W25Q\_ReadData} permet de lire un certain nombre d'octets
dans la mémoire flash à partir d'une adresse donnée. Elle envoie la commande de
lecture de données à la mémoire flash. A noter que la tâche attend d'abord que le
booléen \texttt{ASYNC\_busy} de la structure \texttt{W25Q\_Chip} soit faux avant de
s'executer proprement et le met ensuite à vrai.

\begin{lstlisting}[style=prog, frame=shadowbox, label={lst:ASYNC_W25Q_ReadData},
    emph={[1]ASYNC_W25Q_ReadData_init, ASYNC_W25Q_ReadData}, emphstyle={[1]\color{C}},
    emph={[2]W25Q_Chip, SCHEDULER, TASK}, emphstyle={[2]\color{E}}]
void ASYNC_W25Q_ReadData_init(TASK      *task,
                              W25Q_Chip *w25q_chip,
                              uint8_t   *data,
                              uint32_t   addr,
                              uint32_t   data_size);
void ASYNC_W25Q_ReadData(SCHEDULER *scheduler, TASK *self);
\end{lstlisting}

Le paramètre \texttt{data} est un pointeur vers un tableau d'octets qui contiendra
les données lues. Le paramètre \texttt{addr} est l'adresse à partir de laquelle les
données seront lues. Le paramètre \texttt{data\_size} est le nombre d'octets à lire.
La tâche requiert que le tableau \texttt{data} soit alloué, que sa taille soit
suffisante pour contenir les données lues et qu'il reste en mémoire jusqu'à la fin
de l'exécution de la tâche. La tâche se termine lorsque l'envoie de la commande de lecture s'est bien déroulé.
Elle mettra alors le booléen \texttt{ASYNC\_busy} à faux avant de disparaitre.


\subsubsubsection*{\texttt{ASYNC\_W25Q\_PageProgram} : }

La tâche \texttt{ASYNC\_W25Q\_PageProgram} permet d'écrire un certain nombre d'octets
dans la mémoire flash à partir d'une adresse donnée. L'opération d'écriture ne permet
d'écrire que sur des données préalablement effacées et seulement sur une page de 256
octets. Si l'utilisateur tente d'écrire plus de données que la taille restante de la
page, la tâche écrira uniquement les données qui peuvent tenir dans la page. A noter
que la tâche attend d'abord que le booléen \texttt{ASYNC\_busy} de la structure
\texttt{W25Q\_Chip} soit faux avant de s'executer proprement et le met ensuite à vrai.

\begin{lstlisting}[style=prog, frame=shadowbox, label={lst:ASYNC_W25Q_PageProgram},
    emph={[1]ASYNC_W25Q_PageProgram_init, ASYNC_W25Q_PageProgram}, emphstyle={[1]\color{C}},
    emph={[2]W25Q_Chip, SCHEDULER, TASK}, emphstyle={[2]\color{E}}]
void ASYNC_W25Q_PageProgram_init(TASK       *task,
                                 W25Q_Chip  *w25q_chip,
                                 uint8_t    *data,
                                 uint32_t    addr,
                                 uint16_t    data_size);
void ASYNC_W25Q_PageProgram(SCHEDULER *scheduler, TASK *self);
\end{lstlisting}

Le paramètre \texttt{data} est un pointeur vers un tableau d'octets qui contient les
données à écrire. Le paramètre \texttt{addr} est l'adresse à partir de laquelle les
données seront écrite. Le paramètre \texttt{data\_size} est le nombre d'octets à
écrire. La tâche effectue une copie des données à écrire dans un tableau interne
au moment de son initialisation. Il n'est donc pas nécessaire de conserver le tableau
\texttt{data} en mémoire jusqu'à la fin de l'exécution de la tâche. La tâche se
termine lorsque l'envoie de la commande d'écriture s'est bien déroulé. Elle mettra
alors le booléen \texttt{ASYNC\_busy} à faux avant de disparaitre.

\subsubsubsection*{\texttt{ASYNC\_W25Q\_WriteData} : }

La tâche \texttt{ASYNC\_W25Q\_WriteData} permet d'écrire un certain nombre d'octets
dans la mémoire flash à partir d'une adresse donnée en permettant d'écrire sur
plusieurs pages. 

\begin{lstlisting}[style=prog, frame=shadowbox, label={lst:ASYNC_W25Q_WriteData},
    emph={[1]ASYNC_W25Q_WriteData_init, ASYNC_W25Q_WriteData}, emphstyle={[1]\color{C}},
    emph={[2]W25Q_Chip, SCHEDULER, TASK}, emphstyle={[2]\color{E}}]
void ASYNC_W25Q_WriteData_init(TASK      *task,
                               W25Q_Chip *w25q_chip,
                               uint8_t   *data_buf,
                               uint32_t   addr,
                               uint32_t   data_size);
void ASYNC_W25Q_WriteData(SCHEDULER *scheduler, TASK *self);
\end{lstlisting}

Le paramètre \texttt{data\_buf} est un pointeur vers un tableau d'octets qui contient
les données à écrire. Le paramètre \texttt{addr} est l'adresse à partir de laquelle
les données seront écrite. Le paramètre \texttt{data\_size} est le nombre d'octets à
écrire. La tâche effectue une copie des données à écrire dans un tableau interne
au moment de son initialisation. Il n'est donc pas nécessaire de conserver le tableau
\texttt{data\_buf} en mémoire jusqu'à la fin de l'exécution de la tâche. La tâche
découpe les données à écrire enplusieurs pages de 256 octets et appelle la tâche
\texttt{ASYNC\_W25Q\_PageProgram} pour écrire dans chacune des pages. La tâche se
termine lorsque toutes les pages ont été écrites. [SCHEMA NECESSAIRE ?]

\subsubsection{Exemple d'utilisation}

Illustrons l'utilisation de la mémoire flash W25Q\_XXX avec un exemple simple. Dans
cet exemple, nous allons simplement écrire et lire des données dans la mémoire flash.

\begin{lstlisting}[style=prog, frame=shadowbox, caption={Exemple d'utilisation de la mémoire flash W25Q\_XXX}, label={lst:w25q_example},
    emph={[1]main, W25Q_Init, ASYNC_W25Q_WriteData_init, ASYNC_W25Q_WriteData, ASYNC_W25Q_ReadData_init, ASYNC_W25Q_ReadData, init_scheduler, add_task, run_scheduler}, emphstyle={[1]\color{C}},
    emph={[2]STATUS, W25Q_Chip, SCHEDULER, TASK}, emphstyle={[2]\color{E}}]
#include "w25q_mem.h"
#include "scheduler.h"


typedef enum STATUS {
    INIT,
    WRITE,
    READ,
    STOP
} STATUS;


void main() {
    W25Q_Chip w25q_chip;
    W25Q_Init(&w25q_chip, &hspi1_flag, GPIOA, GPIO_PIN_4, W25Q_ID);
    
    SCHEDULER scheduler;
    init_scheduler(&scheduler);

    STATUS status = INIT;

    bool is_done = false;

    uint8_t data_write[256*4] = {...};
    uint8_t data_read[256*5] = {0};


    while (1) {
        switch (status) {
        case INIT: {
            TASK *task_write = add_task(&scheduler, ASYNC_W25Q_WriteData);
            ASYNC_W25Q_WriteData_init(task_write, &w25q_chip,
                                      data_write, 0, 256*4);
            task_write->is_done = &is_done;
            status = WRITE;
            break;}
        case WRITE: {
            if (is_done) {
                is_done = false;
                TASK *task_read = add_task(&scheduler, ASYNC_W25Q_ReadData);
                ASYNC_W25Q_ReadData_init(task_read, &w25q_chip,
                                         data_read, 0, 256*5);
                task_read->is_done = &is_done;
                status = READ;
            }
            break;}
        case READ: {
            if (task_read->is_done) {
                // Can do something with data\_read here
                status = STOP;
            }
            break;}
        case STOP: { break; }
        }
        run_scheduler(&scheduler);
    }
}
\end{lstlisting}



\begin{minipage}{0.95\textwidth}
\begin{wrapfigure}{L}{0.45\textwidth}
\vspace{-0.8cm}
\dirtree{%
    .0 .
    .1 ASYNC\_W25Q\_WriteData.
        .2 ASYNC\_W25Q\_PageProgram.
            .3 ASYNC\_W25Q\_WaitForReady.
                .4 ASYNC\_W25Q\_ReadStatusReg.
                    .5 ASYNC\_SPI\_TxRx\_DMA.
                    .5 ASYNC\_SPI\_TxRx\_DMA.
                    .5 ASYNC\_SPI\_TxRx\_DMA.
            .3 ASYNC\_W25Q\_WriteEnable.
                .4 ASYNC\_SPI\_TxRx\_DMA.
            .3 ASYNC\_SPI\_TxRx\_DMA.
        .2 ASYNC\_W25Q\_PageProgram (répété par le nbr de page à écrire).
    .1 ASYNC\_W25Q\_ReadData.
        .2 ASYNC\_W25Q\_WaitForReady.
            .3 ASYNC\_W25Q\_ReadStatusReg.
                .4 ASYNC\_SPI\_TxRx\_DMA.
                .4 ASYNC\_SPI\_TxRx\_DMA.
                .4 ASYNC\_SPI\_TxRx\_DMA.
        .2 ASYNC\_SPI\_TxRx\_DMA.
}
\caption{Arboressance des tâches pour l'exemple \ref{lst:w25q_example}}
\end{wrapfigure}

Ce programme est une machine à état qui écrit des données dans la mémoire flash,
puis les lit et enfin s'arrête. La fonction \texttt{main} initialise la mémoire
flash et le scheduler. Elle crée ensuite une tâche pour écrire des données dans la
mémoire flash. Lorsque la tâche est terminée, elle crée une tâche pour lire les
données écrites. Lorsque la tâche de lecture est terminée, le programme ne fait
plus rien.

\vspace{0.5cm}

L'arboressance des tâches pour cet exemple est représentée dans le schéma ci-contre.
On peut voir alors que l'écriture de données dans la mémoire flash est une opération
complexe qui nécessite plusieurs tâches pour s'executer. En effet, l'appelle
à la tâche \texttt{ASYNC\_W25Q\_PageProgram} entraine la génération de 8 sous-tâches.
On peut donc voir l'importance de la gestion des tâches asynchrones et la nécessité
d'avoir une pleine conscience de l'arboressance des tâches pour éviter d'être à cours
de ressources.

\end{minipage}
