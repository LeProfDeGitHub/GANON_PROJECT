\subsection{Flash Stream}
\label{subsec:flash_stream}

Le module \texttt{flash\_stream} est un composant logiciel permettant de gérer la
lecture et l'écriture de données dans une mémoire flash tel que la W25Q tout en
gérant les pointeurs de lecture et d'écriture.\\

Les fichiers correspondants à ce module sont les suivants:
\begin{itemize}
    \item \texttt{flash\_stream.h}
    \item \texttt{flash\_stream.c}
\end{itemize}

\subsubsection{Déclaration des structures et des fonctions}

\subsubsubsection*{\texttt{FLASH\_STREAM}}

La structure \texttt{FLASH\_STREAM} contient un pointeur vers un composant W25Q
ainsi que les pointeurs de lecture et d'écriture.\\

\begin{lstlisting}[style=prog, frame=shadowbox, label={lst:fs_struct},
    emph={[1]flash_stream_init}, emphstyle={[1]\color{C}},
    emph={[2]FLASH_STREAM, W25Q_Chip}, emphstyle={[2]\color{E}}]
typedef struct FLASH_STREAM {
    W25Q_Chip   *flash_chip;

    uint32_t     write_ptr;
    uint32_t     read_ptr;
} FLASH_STREAM;

void flash_stream_init(FLASH_STREAM* stream, W25Q_Chip* flash_chip);
\end{lstlisting}

Les pointeurs de lecture et d'écriture sont initialisés à 0 lors de l'appel de la
fonction \texttt{flash\_stream\_init}.

\subsubsubsection*{\texttt{ASYNC\_fs\_read} et \texttt{ASYNC\_fs\_write}}

Les tâches \texttt{ASYNC\_fs\_read} et \texttt{ASYNC\_fs\_write} permettent
respectivement de lire et d'écrire des données dans une mémoire flash. Leur
contexte étant similaire, ils sont initialisés par la même fonction
\texttt{ASYNC\_fs\_read\_write\_init}.\\

\begin{lstlisting}[style=prog, frame=shadowbox, label={lst:fs_rw},
    emph={[1]ASYNC_fs_read_write_init, ASYNC_fs_read, ASYNC_fs_write}, emphstyle={[1]\color{C}},
    emph={[2]FLASH_STREAM, TASK, SCHEDULER}, emphstyle={[2]\color{E}}]
void ASYNC_fs_read_write_init(TASK *task, FLASH_STREAM *stream,
                              uint8_t *data, uint16_t len);
void ASYNC_fs_read(SCHEDULER *scheduler, TASK *self);
void ASYNC_fs_write(SCHEDULER *scheduler, TASK *self);
\end{lstlisting}

Une fois appelées pour la première fois, les tâches \texttt{ASYNC\_fs\_read} et
\texttt{ASYNC\_fs\_write} incrémentent respectivement les pointeurs de lecture et
d'écriture de la taille des données lues ou écrites. Elle appellent ensuite la
tâche \texttt{ASYNC\_W25Q\_ReadData} ou \texttt{ASYNC\_W25Q\_WriteData} puis se
terminent sans attendre la fin de l'opération. Cependant, elles transmettent à
la tâche précédament générée le pointeur vers le booléen \texttt{self->is\_done}
ce qui permet à la tâche appelante de détecter la fin de l'opération. Puisque aucune
copie de données n'est effectuée, le tableau \texttt{data} doit rester valide
jusqu'à la fin de l'opération.

\newpage

\subsubsubsection*{\texttt{ASYNC\_fs\_read\_floats} et \texttt{ASYNC\_fs\_write\_floats}}

Les tâches \texttt{ASYNC\_fs\_read\_floats} et \texttt{ASYNC\_fs\_write\_floats}
sont des versions des tâches \texttt{ASYNC\_fs\_read} et \texttt{ASYNC\_fs\_write}
spécialisées pour la lecture et l'écriture de données de type \texttt{float}. Tous
comme ces dernières, elles sont initialisées par la même fonction
\texttt{ASYNC\_fs\_read\_write\_floats\_init}.\\

\begin{lstlisting}[style=prog, frame=shadowbox, label={lst:fs_rw_floats},
    emph={[1]ASYNC_fs_read_write_floats_init, ASYNC_fs_write_floats, ASYNC_fs_read_floats}, emphstyle={[1]\color{C}},
    emph={[2]FLASH_STREAM, TASK, SCHEDULER}, emphstyle={[2]\color{E}}]
void ASYNC_fs_read_write_floats_init(TASK *task, FLASH_STREAM *stream,
                                     float *data, uint16_t len);
void ASYNC_fs_read_floats(SCHEDULER *scheduler, TASK *self);
void ASYNC_fs_write_floats(SCHEDULER *scheduler, TASK *self);
\end{lstlisting}

Du fait de la conversion des données de type \texttt{float} en tableau de type
\texttt{uint8\_t}, les tâches \texttt{ASYNC\_fs\_read\_floats} et
\texttt{ASYNC\_fs\_write\_floats} nécessitent l'utilisation de buffers internes
pour stocker les données converties. Ces buffers sont alloués dynamiquement lors
de l'appel de la fonction d'initialisation et libérés à la fin de l'opération.
Contrairement aux tâches \texttt{ASYNC\_fs\_read} et \texttt{ASYNC\_fs\_write},
les tâches \texttt{ASYNC\_fs\_read\_floats} et \texttt{ASYNC\_fs\_write\_floats}
restent actives jusqu'à la fin de l'opération (nécessité de maintenir les buffers
internes existant).\\

[TO BE CONTINUED...]
