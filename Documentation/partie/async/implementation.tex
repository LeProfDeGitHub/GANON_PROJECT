\subsection{Implémentation}
\label{subsec:implementation}

Ci-dessous est présenté le code source de la déclaration des fonctions principales associées à l'ordonanceur
et aux tâches. L'implémentation même de ces fonctions n'est pas présentée ici puisque jugé trop longue et
non essentielle à la compréhension du concept. L'implémentation complète est disponible dans le code source
du projet.

\begin{lstlisting}[style=prog, frame=shadowbox, caption={Définition des tâches et de l'ordonanceur (scheduler.h)}, label={lst:async_h},
    emph={[1]ASYNC_CONTEXT_DEFAULT_BYTES_SIZE, MAX_TASKS_NBR, init_scheduler, add_task, kill_task, run_task,
    run_scheduler}, emphstyle={[1]\color{C}},
    emph={[2]SCHEDULER, TASK, __task_func_t, task_func_t}, emphstyle={[2]\color{E}}]
#include <stdbool.h>
#include "stm32f4xx_hal.h"

#define ASYNC_CONTEXT_DEFAULT_BYTES_SIZE 64
#define MAX_TASKS_NBR 50

// fonction a executer (prend en param le scheduler et le contexte)
typedef void(*__task_func_t)(void*, void*);

typedef struct {
    __task_func_t   func;
    size_t          idx;
    bool           *is_done;
    uint8_t         context[ASYNC_CONTEXT_DEFAULT_BYTES_SIZE];
} TASK;

typedef struct {
    size_t   length;
    size_t   current_idx;
    size_t   nbr_has_run_task;
    TASK    *tasks[MAX_TASKS_NBR];
    bool     running[MAX_TASKS_NBR];
} SCHEDULER;

typedef void(*task_func_t)(SCHEDULER*, TASK*);


// Initialise l'ordonanceur notamment en allouant la zone mémoire pour les tâches.
void init_scheduler(SCHEDULER *scheduler);

// Initialise une tâche, la place dans la zone mémoire allouée, l'ajoute à la liste
// des tâches, met à jour le tableau running (active la tâche), les variables de
// l'ordonanceur et renvoie l'addresse mémoire de la tâche.
TASK *add_task(SCHEDULER *scheduler, task_func_t func);

// Met à jour le tableau running (désactive la tâche), les variables de l'ordonanceur
// et le drapeau "*is\_done" de la tâche s'il est défini.
void kill_task(SCHEDULER *scheduler, TASK *task);

// Exécute une tâche
void run_task(SCHEDULER *scheduler, TASK *task);

// Exécute la tâche suivante
void run_scheduler(SCHEDULER *scheduler);
\end{lstlisting}