\subsection{Une tâche (TASK)}
\label{subsec:task}

Contrairement à une fonction classique, une tâche est une fonction qui peut commencer,
interrompre et reprendre son exécution en plusieurs temps. Dans l'OMEGAAA, une tâche se
matérialise par une structure de données contenant trois éléments : une fonction classique
(\texttt{func}\footnote{\label{footnote:ref_async_h}cf Code \ref{lst:async_h}}), un indice
de référence(\texttt{idx}\footnoteref{footnote:ref_async_h}) et une sous-structure de données
appelée "contexte d'exécution" (\texttt{context}\footnoteref{footnote:ref_async_h}).
Le contexte d'exécution contient les informations nécessaires pour reprendre l'exécution de
la tâche là où elle s'était arrêtée auparavant.

Par simplicité et pour des raisons de performances, bien que chaque type de tâche a besoin
d'un contexte d'exécution unique, \texttt{context} est une zone mémoire de taille fixe.
Cette taille est définie par la constante \texttt{ASYNC\_CONTEXT\_DEFAULT\_BYTES\_SIZE}
\footnoteref{footnote:ref_async_h} et est par défaut de 64 octets. Cela permet d'uniformiser
la structure de toutes les tâches. Cette taille doit être suffisante pour contenir le contexte
d'exécution de n'importe quelle tâche et doit être redéfini en fonction des besoin du programme
tout en considérant les limites de mémoire du microcontrôleur.