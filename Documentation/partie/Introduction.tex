% \setcounter{page}{1}
\section*{Mise en contexte}
\addcontentsline{toc}{section}{\protect\numberline{}Mise en contexte}

Ce document a pour but de présenter l'OMEGAAA, un concepte d'ordinateur de bord
pour une fusée amateur. Le projet a été mené par Alexis Paillard et se base en
majeur partie sur le projet Unknown réalisé dans le cadre de l'association
AeroIPSA.\\

\subsection*{- AeroIPSA}
AéroIPSA est une association étudiante de l'école d'ingénieurs IPSA qui conçoit et
réalise entièrement des projets fonctionnels en rapport avec le secteur
aérospatial. Elle rassemble des étudiants autour de projets d'astromodélisme,
principalement de lanceurs, mais aussi de Cansats (micro-satellites atmosphérique).
Cela permet aux différents membres de l'association d'appliquer les notions
apprises durant leur cursus au profit d'un projet d'envergure et d'acquérir les
compétences nécessaires dans leur futur métier d'ingénieur.\\

\subsection*{- Unknown}
Unknown est un projet de module électronique de fusée expérimentale réalisé par
Vincent Fauquembergue et Alexis Paillard. Les expériences du projet Unknown ont été
les suivantes :
\begin{itemize}
    \item \textbf{Expérience principale} : Relocaliser une fusée après son lancement
    grâce à un module de télémesure LoRa renvoyant les données GNSS tout au long du
    vol.
    \item \textbf{Expérience secondaire} : Réalisation d'une collecte de données
    provenant de nombreux capteurs, barométrique et centrale inertielle, afin de
    reconstituer le vol après récupération des données stockées sur une mémoire
    flash.
\end{itemize}
L'un des objectifs du projet Unknown était de réaliser un module électronique se
voyant plus facilement intégrable dans une fusée amateur. Cela s'est traduit par
l'utilisation d'une seule carte électronique ayant tous ses composants directement
soudés dessus ainsi que l'utilisation d'un microcontrôleur autre que l'Arduino ou
que la Teensy. Le choix fait s'est porté sur un STM32F4 de STMicroelectronics. La
programmation de ce microcontrôleur a été réalisée en C et un bon nombre des
drivers nécessaires ont dû être réadapté par les membres du projet ce qui a permis
d'acquérir de nouvelles compétences en programmation bas niveau.

L'annexe \ref{anx:unknown} présente le rapport du projet Unknown ainsi que des
photos du module.

\subsection*{- OMEGAAA}
OMEGAAA est l'acronyme de \textbf{O}rdinateur de \textbf{M}esure,
d'\textbf{E}valuation, de \textbf{G}estion et d'\textbf{A}nalyse pour
l'\textbf{A}éro-nautique et l'\textbf{A}érospatiale. Il s'agit d'un concepte
d'ordinateur de bord qui a pour but de gérer les différentes phases de vol d'une
fusée amateur, tout en effectuant des mesures notamment d'accélérations et de
vitesses de rotation (à l'aide d'un IMU). Ces données sont ensuite traitées pour en
extraire la position et l'attitude de la fusée et enregistrées dans une mémoire.
L'OMEGAA gère également une communication par télémétrie avec une station au sol.

Ce module fait suite au projet Unknown en reprenant les composants électroniques de
ce dernier. Il a pour but de continuer à développer les compétences en
programmation acquises lors du projet Unknown et d'en développer de nouvelle comme
la création du RTOS \footnote{Un RTOS (\textbf{R}eal \textbf{T}ime
\textbf{O}perating \textbf{S}ystem)  est un système d'exploitation conçu pour gérer
les ressources matérielles de manière à garantir que des tâches critiques soient
exécutées parallèlement à d'autre et dans des délais stricts et prévisibles.
Ce type de système est laregement employé dans des application tels que les systèmes
embarqués, les dispositifs médicaux, les systèmes de contrôle industriel, et les
applications aéronautiques et aérospatiales...}. Cet ordinateur de bord étant
fictif, il n'a pas été réalisé mais l'ensemble des travaux et des recherches
effectuées pour sa conception sont présentées dans ce document et ont pour but de
servir de support pour la réalisation d'un tel module électronique.\\
